\documentclass[a4paper]{article}
\usepackage[utf8]{inputenc}
\usepackage[english]{babel}
\usepackage[T1]{fontenc}
%\usepackage[pdftex]{color,graphicx}
%\usepackage{colortbl}
%\usepackage{pdfpages}
\usepackage{fancyhdr}
%\usepackage{amsmath} % math environment
%\usepackage{listings}
\usepackage{xcolor, colortbl}
\usepackage{graphicx}
\usepackage{pdfpages}
\usepackage{setspace} % text spacing options
\usepackage{textcomp} % text symbols
\usepackage{multicol}
\usepackage{calc}
\usepackage{rotating}

%temp
\usepackage{lipsum}

\pagestyle{fancy}

%Define commands
\newcommand{\systemname}{Jenkins validated merging}
\newcommand{\groupname}{Team $\Delta$}
\newcommand{\groupmembers}{
	Andreas Frisch \{andreas.frisch@gmail.com\}, \\
	Esben Skaarup \{esben.skaarup@gmail.com\}, \\
	Alexander W. Uldall \{morpmex@gmail.com\} \\
	Ronni Elken Lindsgaard \{ronni.lindsgaard@gmail.com\}, \\
	~
}

%Define colors
\definecolor{Gray}{gray}{0.8}
\definecolor{DarkGray}{gray}{0.4}

\lhead{Project Course: Development Studio}
\chead{}
\rhead{\systemname}
\lfoot{}
\cfoot{\thepage}
\rfoot{}

%Start section numbering from 0
\setcounter{section}{-1}

\begin{document}
\begin{titlepage}
	% Title
	\begin{center}
		\vspace*{4cm}
		\rule{\linewidth}{0.5mm}\\[0.4cm]
		{\huge \bfseries \systemname}
		\rule{\linewidth}{0.5mm}
	\end{center}
	\begin{flushleft}
		{
			\Large Project Course: Development Studio \\[0.1cm]
			{\it Assignment 2}
		}
	\end{flushleft}
	\vspace*{4cm}
	
	% Authors
	\begin{flushleft}
		{\Large \groupname :} \\[0.1cm]
		{\Large \groupmembers} \\[0.3cm]
		{\Large \today}
	\end{flushleft}
\end{titlepage}
\newpage
\onehalfspacing
\setcounter{tocdepth}{2}
%\tableofcontents
%newpage

\section{This hand-in}
This assignment is written as part of the course {\it Project Course: Development Studio} at DIKU.

This assignment concerns itself with our results from the first sprint.
Furthermore we will briefly discuss the lessons learned and how they affect our
upcoming sprint.

Our customer, Praqma, wants a plug-in for the continuous delivery facilitator
Jenkins, which allows Jenkins to rollback broken commits, thus maintaining an
unpolluted company truth repository.

Our code can be found at https://github.com/pcds2013-team-delta/pretest-commit-plugin

\section{Sprint results}
Being unfamiliar with both Jenkins, Maven, continuous delivery and to some extend
Mercurial, we spent a lot of the first sprint setting up environments and
familiarizing ourselves with the various tools. We did this in agreement with
the customer, whose only coding demands were that we created a footprint in
Jenkins and set up a work pipeline.

However, having research tasks without any visible deliverable is a recipe for
disaster, which Praqma already knows. They asked us to describe tangible
deliverables for all tasks, research as well as implementation.

As such the sprint resulted in both documentation (which can be seen on our Wiki
page
\footnote{https://github.com/pcds2013-team-delta/pretest-commit-plugin/wiki/\_pages})
and code (which can be found in our github
repository\footnote{https://github.com/pcds2013-team-delta/pretest-commit-plugin}).

\section{Sprint review and discussion}
We did not get around to do all tasks in our sprint backlog. However, we divided
the tasks into categories {\it must have}, {\it should have} and
{\it could have} before the sprint began. Our aim was to complete all
{\it must haves}, hopefully get around to resolve all {\it should
haves}, but only consider {\it could haves} if time permitted.

This is a bit different from a standard SCRUM approach, but as we are working on
new tools using new methods (story points and SCRUM) we decided to work in a bit
of slack. In case we grossly underestimated our own ability, we had the {\it
could haves} to do and in case of gross overestimation of our ability our
{\it must haves} only cover around two thirds of the overall sprint
capacity. This way is inspired by Praqma's pragmatic SCRUM approach and fits nicely into
our lines of thought, why we decided to continue doing this.

We presented our results to the customer, demonstrating footprints and
documentation. The customer agreed that the time spent researching our tools and methods was
time well spent and was generally positive regarding our progress.

Based on the completed amount of story points in contrast to our estimated
velocity during the previous sprint, we calculated our efficiency, which we used
to calculate our coming sprint capacity. This value will change from sprint to
sprint. However, we forgot to measure our story point precision (how many story
points we spend per estimated story point). This last bit is important to
properly estimate sprint capacity in the future and will be included from now on
out.

\subsection{Story point estimation}
For our first sprint we estimated our sprint capacity to $50$. We completed $35$
story points worth of tasks, yielding an efficiency of
$$35/50 = 0.7$$

For this second sprint, which consists of more weeks than the first sprint, we
estimated a $83$ story points capacity. This takes into consideration that one
of the weeks is a holiday and part of the group has exams during this sprint.

Granted our current efficiency we have
$$83*0.7 = 58$$
story points worth of actual capacity for the coming sprint.

\section{Upcoming Sprint}
In collaboration with the customer, we determined the goal and tasks for the
next sprint and generated the sprint backlog. This sprint has a much more
intensified focus on actual implementation and possible visual presentation of
functionality than the first. As such we are going to write more tests this
sprint, why we now introduce the {\textit tester} role defined in the last
report.

This role was absent from the currently finished sprint, as we had little code
to possibly apply any tests to.

\end{document}
